\chapter{Einleitung}
\label{cha:Einleitung}

In diesem Dokument werden einige Beispiele gegeben, die Ihnen das Erstellen Ihrer Arbeit erleichtern soll. Wir geben Ihnen hier Tipps zum Umgang mit \LaTeX, aber auch einige Hinweise zum Erstellen wissenschaftlicher Arbeiten. 

Wenn Sie ein neues Kapitel beginnen, beachten Sie bitte, dass dieses ebenfalls eine Einleitung aufweist. Das heißt, dass Kapitelüberschriften \textit{nicht} für sich alleine stehen und beispielsweise direkt die Überschrift des ersten Abschnittes folgt.


\section{Tabellen}
\label{sec:Tabellen}
Dieser Abschnitt gibt Beispiele für die Verwendung von Tabellen.

\begin{table}[ht]
	\centering
    \caption[Kurztitel Tabelle]{Hier steht der lange Titel für die Tabelle}
    \vspace{1.0em}	
	\begin{tabular}{|l|r|}
\hline
Text & 12\% \\
\hline
Text & 34\% \\
\hline
Text & 56\% \\
\hline
Text & 78\% \\
\hline
Text & 90\% \\
\hline
		\end{tabular}
	\vspace{1.0em}
	\label{tab:tabelle}
\end{table}

\noindent{}Dies ist lediglich ein Beispiel. Je nach beabsichtigter Aussage, können Tabellen ganz unterschiedlich aussehen. Ein weiteres Beispiel:

\begin{table}[ht]
	\centering
	\caption[Alternative Tabelle]{Alternative Tabelle mit farbiger Kopfzeile}
		\vspace{1.0em}	
	\begin{tabular}{|c|l|}
		\hline
		\rowcolor[gray]{0.9}\textbf{numbers} & \textbf{text} \\
		\hline
		\hline
		1 & This text flush-left \\
		\hline
		2 & while the numbers are \\
		\hline
		3 & centred \\
		\hline
	\end{tabular}
	\label{tab:tablealternative}
\end{table}

Bitte beachten Sie: Tabellen haben in der Regel \textit{Über}schriften, während Abbildungen \textit{Unter}schriften aufweisen. Im Quellcode sehen Sie, dass im "caption" ein kurzer Titel vergeben wird, der für das Tabellenverzeichnis vergeben wird. Der längere Titel wird als Überschrift verwendet.  \\

Das Erstellen von Tabellen kann sehr aufwändig sein. Die folgenden Werkzeuge können hier sehr hilfreich sein:

\begin{itemize}
	\item Excel to \LaTeX{} Converter\\ \url{https://github.com/krlmlr/Excel2LaTeX/releases}
	\item Apple Script: Numbers to \LaTeX{} \\ \url{https://gist.github.com/pgundlach/386384}
	\item Gnumeric (hat eine Export-Funktion für \LaTeX{}): \\ \url{https://projects.gnome.org/gnumeric/}
	\item OpenOffice, Calc2LaTeX: \url{http://extensions.openoffice.org/de/project/calc2latex-macro-converting-openofficeorg-calc-spreadsheets-latex-tables}
\end{itemize}

\section{Tabellen referenzieren}
\label{sec:tabellen_ref}
In diesem Abschnitt werden die Tabellen aus dem vorigen Abschnitt im Text referenziert. Dies ist der Bezug auf die erst Tabelle: \ref{tab:tabelle}, und hier der Verweis auf die zweite Tabelle \ref{tab:tablealternative}. Vergeben Sie am besten immer jeder Tabelle, Abbildung, Abschnitt, Kapitel, etc. ein individuelles label, so dass Sie dieses dann zum Referenzieren verwenden können. Im Übrigen stehen Abbildungen und Tabellen niemals für sich alleine, sondern sollten im Text diskutiert werden und somit natürlich auch referenziert. Die Verwendung des "ref{}"-Befehls sorgt dafür, dass immer auf die richtige Nummerierung verwiesen wird, selbst wenn der Text später geändert wird (z.B., wenn Tabellen hinzugefügt oder gelöscht werden). 

\section{Anführungszeichen}
\label{sec:Anfuehrungszeichen}

Es gibt verschiedene Optionen, Anführungszeichen zu generieren. Da das entsprechende Paket eingebunden wurde, können Sie folgende Optionen verwenden:

\begin{itemize}
	\item \enquote{Befehl des Pakets: csquotes}
	\item Alternativ: ``example''
	\item Oder "Beispiel". 
\end{itemize}

Wie Ihnen vielleicht aufgefallen ist, sehen die Alternative etwas anders aus. Für die Verwendung von Anführungszeichen gibt es Konventionen, siehe \url{https://de.wikibooks.org/wiki/LaTeX-W%C3%B6rterbuch:_Anf%C3%BChrungszeichen}






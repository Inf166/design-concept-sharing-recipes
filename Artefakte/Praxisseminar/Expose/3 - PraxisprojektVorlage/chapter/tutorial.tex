\chapter{Zusammenfassungen von Joel}
\label{cha:joel}

\section{Optionen} \label{sec:options}
    Einrückungen entfernen
    \texttt{ \textbackslash setlength\{parindent\}\{0pt\} } 
    \setlength{parindent}{0pt}

\section{Anführungszeichen} \label{sec:marks}    
    \texttt{ \textbackslash enquote\{Text\} } \\
    \enquote{Text} \\
    
    \texttt{ \textbackslash glqq Text \textbackslash grqq } \\
    \glqq Text \grqq \\

\section{Zitieren} \label{sec:quotes}
    \texttt{ \textbackslash cite\{Dix04\} } \\
    Gibt nur den Autornamen und Jahr in Klammern aus \\
    \cite{Dix04} \\

    \texttt{ \textbackslash cite[S. 86]\{Dix04\} } \\
    Zitat komplett in Klammern mit Komma getrennt \\
    \cite[S. 86]{Dix04} \\
    
    \texttt{ \textbackslash citep\{Dix04\} } \\
    Klammern um Autorname und Jahr \\
    \citep{Dix04} \\

    \texttt{ \textbackslash citep[S. 86]\{Dix04\} } \\
    Zitat mit Klammer außen rum: \\
    \citep[S. 86]{Dix04} \\
 
    \texttt{ \textbackslash citet\{Dix04\} } \\
    Gibt nur den Autornamen und Jahr in Klammern aus \\
    \citet{Dix04} \\

    \texttt{ \textbackslash citet[S. 86]\{Dix04\} } \\
    Zitat mit Klammern um [] plus Jahr, aber Autor ohne Klammern \\
    \citet[S. 86]{Dix04} \\

    \texttt{ \textbackslash citeauthor\{Dix04\} } \\
    Gibt nur den Namen ohne Klammern und so weiter \\
    \citeauthor{Dix04} \\
    
    \texttt{ \textbackslash citep[vgl.][Seite 1]\{Dix04\} } \\
    Liefert erste eckige Klammer plus Namen mit Komma separierten Jahr und zweiter eckiger Klammer. Alles in Klammern. \\
    \citep[vgl.][Seite 1]{Dix04} \\

\section{Fußnote} \label{sec:notes}
    \texttt{ \textbackslash footnote\{ American Psychological Association (APA) \} } \\
    \footnote{ American Psychological Association (APA) }

\section{Referenzieren} \label{sec:refs}
    \texttt{ \textbackslash ref\{ sec:refs \} } \\
    Liefert die Sprungmarke mit Nummer des Chapters, Section oder Figure ...
    \ref{sec:refs}

    \texttt{ \textbackslash pageref\{ sec:refs \} } \\
    Liefert die Sprungmarke mit Seitenzahl
    \pageref{sec:marks}

\section{Formatierung} \label{sec:format}
    \texttt{ \textbackslash emph\{ Text hier \} } \\
    Kursiv \\
    \emph{Pellentesque habitant morbi tristique senectus} \\

    \texttt{ 
    \textbackslash begin\{ quotation \}
        \textbackslash emph\{ Text hier \}
    \textbackslash end\{ quotation \}
    } \\
    Kursiv und eingerückt \\
    \begin{quotation}
        \emph{Pellentesque habitant morbi tristique senectus}
    \end{quotation} 

    \texttt{ \textbackslash textbf\{ Text hier \} } \\
    Dick gedruckt \\
    Pellentesque habitant morbi tristique \textbf{senectus} \\

    \texttt{ \textbackslash underline\{ Text hier \} } \\
    Unterstrichen \\
    Pellentesque habitant morbi \underline{tristique} senectus \\

    \texttt{ \textbackslash textbf\{ \textbackslash textit\{ Text hier \} \} } \\
    Kursiv und dick gedruckt \\
    \textbf{\textit{Pellentesque}} habitant morbi tristique senectus. \\

\section{Aufzählungen} \label{sec:items}
    Hier folgen einige Beispiele für Listen. Zunächst eine nicht nummerierte Liste: 
    \begin{itemize}
        \item Item 1
        \item Item 2
        \item Item 3
    \end{itemize}

    Nun eine nummerierte Liste:
    \begin{enumerate}
        \item Item 1
        \item Item 2
        \item Item 3
    \end{enumerate}

    Man kann auch Symbole verwenden:
    \begin{itemize}
    \renewcommand{\labelitemi}{$\rightarrow$}
        \item Item 1
        \item Item 2
        \item Item 3
    \end{itemize}

    Oder mit Überschriften
    \begin{description}
        \item[Titel - 1] \hfill \\ 
            Lorem Ipsum dolores et. Itor mene toto hehls.
        \item[Titel - 2] \hfill \\ 
            Lorem Ipsum dolores et. Itor mene toto hehls.
    \end{description}
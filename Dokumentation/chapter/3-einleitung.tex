\chapter{Einleitung}
\label{cha:Einleitung}

\section{Herausforderung und Motivation}
\subsection{Wirtschaftliche Relevanz}
% die Anpassung einer Gestaltung an die Erfordernisse und Fähigkeiten der Benutzer verbessert deren Nutzung, Qualität und Effizienz, wodurch preisgünstige Gestaltungslösungen zur Verfügung gestellt werden und die Wahrscheinlichkeit reduziert wird, dass Systeme, Produkte und Dienstleistungen unwirtschaftlich sind oder von ihren Benutzern abgelehnt werden; 
Bestehende Implementierungen eines Systems, das die beschriebenen Use Cases abdecken soll, mangelt es an Qualität der Benutzbarkeit und erfüllen nicht die erhobenen Anforderungen der Stakeholder. So sind zum Beispiel 52\% der Nutzer sehr unzufrieden mit der größten Rezepteplattform Chefkoch \citep{trustpilot:online}. Software Dienstleistungen werden durch Werbung finanziert \citep[vgl. Punkt 1]{HowDoFre38:online}. Diese Werbung wirkt auf den Nutzer tendenziell ablenkend und wird als störend während des Kochens oder Stöberns wahrgenommen \citep{Allesvol19:online}. Durch die Nutzung von Online Foren und großen internationalen Restaurantketten geraten die lokalen Rezepte der Familie in den Hintergrund \citep{bpb2021fastfood, bpb2021fastfoodtopic, HowHasGl49:online}. Die Implementierung der Rechtefreigaben für eigene Rezepte ist mangelhaft. Es muss gegeben sein, dass Nutzer entscheiden können welche Rezepte öffentlich, mit der Familie geteilt oder privat sind.\\
Hier bedarf es eines Systems, welches die oben gelisteten Punkte adressiert. 

\subsection{Wissenschaftliche Relevanz}
Zur Zeit der Ausarbeitung ist die Etablierung von Progressiven Web Applikationen (kurz PWA) stagniert \citep{Magomadov_2020}. Im Gegensatz zu nativen Applikationen bieten PWAs den Vorteil, dass sie auf jedem Endgerät mit einem Browser installiert werden können, je nach Endgerät und genutztem Browser mit Einschränkungen in den Funktionalitäten \citep{MScthesi20:online, Progress77:online}. \\
Jedoch bringt die Implementierung einer App, ob nativ oder als Web App, für verschiedene Betriebssysteme das Problem mit sich, sodass das User Interface nach den unterschiedlichen Design Guidelines zu gestalten ist \citep{Mitrovic2016ARO}. Hier bedarf es einer Lösung, die auf allen Betriebssystemen von allen Nutzern als ergonomisch bewertet wird. Dementsprechend müssen aktuelle Guidelines untersucht und eine Gestaltungslösung umgesetzt werden, welche die Vorgaben in einem eigenen Styleguide bündelt.\\
Ein System das von Nutzern befüllt und genutzt wird, bedarf besonderer Konzeption in Hinsicht auf Datenschutz und Barrierefreiheit \citep{Privacya9:online}. Im Rahmen der Gestaltung und Architektur soll hier ein Konzept entwickelt werden, welches die Bereiche zufriedenstellend abdeckt.

\subsection{Soziologische Relevanz}
% ein menschzentrierter Ansatz führt zu Systemen, Produkten und Dienstleistungen, die besser für die Gesundheit, das Wohlbefinden und das Engagement ihrer Benutzer sind, einschließlich der Benutzer mit Behinderungen.
Aufgrund der Globalisierung ist die Wahrung und Förderung von Kultur und Tradition eine wichtige Aufgabe \citep{BryanTurner_2016}. Erreicht wird dieses Ziel durch die vereinfachte Weitergabe von Wissen an kommende Generationen. Durch die Dokumentation von Familienrezepten wird aber auch die kritische Auseinandersetzung mit der eigenen Vergangenheit angeregt \citep{76f9434be2ae4e23b9f44d93097c915c}. Zusätzlich werden aus fachlichen Kochbüchern, angereichert durch persönliche Anekdoten, Werke von emotionalem Wert.

\section{Bezug zum Entwicklungsprojekt}
Es wurde bereits im Rahmen des Entwicklungsprojekts \citep{cobanmai2021} ein vertikaler Prototyp entwickelt, welcher die Umsetzbarkeit eines Systems auf Basis der Technologie PWA und Peer to Peer bewertet hat. Die Artefakte dieses Projekts wurden teilweise mit in die Konzeption dieses interaktiven Systems mit eingebracht. Das Einverständnis des Teammitgliedes des Entwicklungsprojekts für die Nutzung der Artefakte liegt vor.

\section{These}
Diese Arbeit setzt sich mit der Gestaltung interaktiver Systeme auseinander, die anhand der DIN-EN-ISO 9421-210 \citep{DINISO_2010} erarbeitet wurden. Es soll festgestellt werden, ob die Qualität der erarbeiteten Gestaltungslösung, für ein System das den Austausch von traditionellen und kulturellen Rezepten ermöglicht, den Nutzererwartungen und Ansprüchen an die Gebrauchstauglichkeit und Ergonomie entspricht.
Es soll festgestellt werden, ob es einen Zusammenhang zwischen der früh durchgeführten Evaluierung und Iteration der Gestaltungslösung und der Zufriedenheit der Nutzer auch im Bezug auf dieses System besteht.
Desweiteren soll festgestellt werden, welchen Einfluss die Auswahl der Nutzergruppe und die Integration dieser auf die Konzeption des Systems haben.

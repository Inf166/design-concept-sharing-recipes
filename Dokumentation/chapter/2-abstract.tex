\chapter*{Kurzfassung}
\addcontentsline{toc}{chapter}{Kurzfassung}
% Basic introduction to the field - verständlich für jeden Wissenschaftler jedes Feldes
Durch die zunehmende Globalisierung erlebt die Gesellschaft den Verlust von Kultur und Tradition. Online Rezepte ersetzen Framilienrezepte. Hier angesetzt, soll durch ein ergonomisches, interaktives System zum Austausch und Weitergabe des Wissens und den Erfahrungen der Familie, das Problem des Kultur- und Traditionsverlusts adressiert und reduziert werden.\\
% 2-3 Sätze zu mehr detaillierten Hintergrund - verständlich für MCIler // Grundsätze der menschzentrierten Gestaltung // einem umfassenden Verständnis der Benutzer // die Benutzer sind während der Gestaltung und Entwicklung einbezogen // Iteration auf der Basis benutzerzentrierter Evaluierung
Um die Weitergabe und den Austausch zu erleichtern, müssen die in diesem Gebiet klassischen Probleme untersucht und gelöst werden. Darunter fallen Probleme wie: der Import und Export, die Einheitlichkeit der Rezepte und damit gesteigerte Lesbarkeit, den Aufbau eines Kochbuchs, die kritische Auseinandersetzung mit Inhalten, aber auch die persönliche Verbindung zu dem gebündelten Wissen. Auf die Anlehung an Kochbücher und gedruckte Rezepte, innerhalb der Gestaltungslösungen wird Wert gelegt. Der Schutz des Kulturguts einer Familie muss durch die Kommunikationswege, Nutzerverwaltung und Zugriffsrechtevergabe behandelt werden. Zusätzlich werden die Barrierefreiheit, der Datenschutz, die Nachhaltigkeit, die Instandhaltbarkeit, die Skalierbarkeit und die Erreichbarkeit untersucht und implementiert.\\
% 1 Satz das Problem adressieren // Bedingungen // Prozess zur Gestaltung gebrauchstauglicher interaktiver Systeme
Die Beteiligung, einer repräsentativen Nutzergruppe, an der Konzeption des Systems und der Priorisierung einzelner Anforderungen ist gewünscht und erforderlich.\\
% 1 Satz das Ergebnis zusammenfassen
Denn hier zeigt sich, dass durch die Zusammenarbeit und die Evaluierung der Stakeholder, die Erfüllungsrate der Akzeptanzkriterien und die Größe der Zielgruppe, positiv beeinflusst wird.\\
% 2-3 Sätze wie das Endergebnis im Vergleich zu bisherigen Wissensstand zu diesem beiträgt/erweitert
Die Anforderung konnten mittels Interviews und gemeinsamer Erarbeitung von User Stories konkretisiert werden. Des Weiteren wurden diese Anforderungen priorisiert und dadurch auch Konflikte zwischen den Anforderungen beseitigt.
Die Evaluierungen der ersten Gestaltungslösungen hat die Gestaltung maßgeblich beeinflusst. Das bestehen des Interesses zeigt die korrekte Wahl des Vorgehens.\\
% 1-2 Sätze wie das Ergebnis generell Einzahlt
Das Endergebnis macht deutlich, dass das gestalten eines Interaktiven Systems nach der DIN-EN-ISO 9241-210 ein wichtiger Teil der Entwicklung sozialer Netzwerke zum Austausch von Wissen und Erfahrungen ist. \\
% 2-3 Sätze zum Anregen weiterer Forschung
Die Entwickler von interaktiven Systemen, sollten sich ebenfalls mit der Zielgruppe befassen und sie in die Gestaltung mit einbeziehen um die Systeme menschenzentrierter zu gestalten. Für die Weiterentwicklung des Datenschutzes sind Forschungen an der PWA und Peer to Peer Technologie notwendig, denn sie würden, wenn sie mehr etabliert und mehr Funktionen bieten, eine noch bessere Alternative darstellen.
\newpage
\chapter*{Abstract}
\addcontentsline{toc}{chapter}{Abstract}
%Hier folgt die Kurzfassung auf Englisch. Wenn Sie diese Vorlage für Seminararbeiten, Projektdokumentation o.ä. verwenden ist eine englische Kurzfassung ggf. nicht nötig.
Due to increasing globalization, society is increasingly experiencing the loss of tradition and culture. Online recipes are replacing family tradition. This is where an ergonomic, interactive system for the exchange and transfer of knowledge and family experience is needed to address and reduce the problem of loss of culture and tradition.
In order to facilitate the transmission and the exchange, it is necessary to study and solve the classical problems in this field. These include problems such as: the import and export, the uniformity of recipes and thus increased readability, the structure of a cookbook, the critical examination of content, but also the personal connection to the pooled knowledge. On the leanings of cookbooks and printed recipes, within the design solutions is emphasized. The protection of a family's cultural assets must be addressed through the communication channels, user management and access rights allocation. Additionally, accessibility, privacy, sustainability, maintainability, scalability, and accessibility will be explored and implemented.
The participation, of a representative user group, in the design of the system and the prioritization of individual requirements is desired and necessary.
This is because it is shown here that through the cooperation and evaluation of the stakeholders, the fulfillment rate of the acceptance criteria and the size of the target group, is positively influenced.
The initial unclear requirements for the system were resolved by means of interviews and the joint development of user stories. In addition, these requirements were prioritized, which also eliminated conflicts between requirements.
The evaluations of the initial design solutions significantly influenced the design. Since there is interest and demand for such a system among the target group after the presentation of the prototypes, it can be deduced that the procedure was chosen correctly and a satisfactory design solution of the MVP was achieved.
The final result shows that the design of an interactive system according to DIN-EN-ISO 9241-210 should be an important part of the development of social networks for the exchange of knowledge and experience. 
Upcoming design solutions and system architectures, should also deal with the target group and include them in the design to make the systems more human-centered. Research on PWA and peer to peer technology is needed for the advancement of privacy, because they would be an even better alternative if they are more established and offer more features.
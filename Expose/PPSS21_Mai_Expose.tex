% PREAMBLE
\documentclass[10pt]{article}

% Packages
\usepackage[usenames]{color}
\usepackage[utf8]{inputenc}
\usepackage{graphicx}
\usepackage{subfig}
\usepackage{url}

% Meta
\author{Joel Mai}
\title{Exposé - Framework zum Synchronisieren und Teilen von Inhalten über Peer to Peer}
\date{30. März 2021}

% CONTENT
\begin{document}

    \maketitle 

    \newpage

    \section{Einleitung}\label{sec:Einleitung}

    \section{Problemraum}\label{sec:Problemraum}
    % Was soll warum behandelt werden?

        \subsection{Domänenmodell}\label{sec:Domaenenmodell}

        \subsection{Problembeschreibung}\label{sec:Problembeschreibung}
        % Existenz und Beschreibung des Problemraums werden durch Quellen belegt

    \section{Zielsetzung und Vision}\label{sec:Zielsetzung}
    % Welchen Mehrwert bringt das Projekt und wie erreiche ich diesen?
    % Die Vision ist in der Regel technologieunabhängig; außer sie sind aus dem Problemraum heraus begründet
    % Was soll warum behandelt werden?

    \section{Relevanz}\label{sec:Relevanz}
    % Inwiefern ist die Adressierung dieser bestimmten Problemstellung mittels dieser bestimmen Zielsetzung relevant?

        \subsection{Gesellschaftliche Relevanz}\label{sec:Gesellschaftliche}
        % gesellschaftliche Relevanz: trägt zur öko-sozialen Transformation bei

        \subsection{Wirtschaftliche Relevanz}\label{sec:Wirtschaftliche}
        % wirtschaftliche Relevanz: z.B. schließt eine Marktlücke, unterstützt innovative Geschäftsmodelle,

        \subsection{Wissenschaftliche Relevanz}\label{sec:Wissenschaftliche}
        % wissenschaftliche Relevanz: bringt einen wissens-orientierten Mehrwert

    \section{Persönliche Motivation}\label{sec:Motivation}
        \cite[Entwicklungsprojekt]{cobanmai2021}  

    % Bibliothek einbinden
    \bibliographystyle{plain} 
    \bibliography{PPSS21_Mai_Expose.bib}

\end{document}
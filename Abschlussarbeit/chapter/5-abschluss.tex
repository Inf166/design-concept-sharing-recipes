\chapter{Fazit}
\label{cha:Fazit}
Anhand der Durchführung des Prozesses zur Gestaltung gebrauchstauglicher interaktiver Systeme - entnommen aus der DIN-EN-ISO 9421-210 \citep{DINISO_2010} - setzte sich diese Arbeit im Wesentlichen mit der Zufriedenstellung der Nutzer mit einem System für den Austausch von traditionellen und kulturellen Rezepten auseinander. Dabei wurden die Schritte aus dem Vorgehensmodell der Norm durchlaufen und fortlaufend evaluiert. Final wurde eine Umfrage durchgeführt, um ein Stimmungsbild der Nutzer zu erfassen und auswerten zu können. \\

Aufgrund der geringen Anzahl an Teilnehmenden an der finalen Evaluierung können keine allgemeingültigen Schlussfolgerungen gezogen werden.
Die Abdeckung der repräsentativen Nutzergruppe für die Evaluierungsprozesse und deren Anzahl ist sehr wichtig, da sich statistische Ausreißer bei Umfragen mit kleineren Gruppen größer auf das Endergebnis auswirken als bei einer größeren Teilnehmerzahl. 
Zu erkennen war die Auswirkung auf das Endergebnis durch größere Teilnehmerzahl, besonders bei der Erarbeitung einer Gestaltungslösung für ein gedrucktes Rezept, um den Styleguide für das System zu erarbeiten (siehe \ref{subsec:styleguide}). Daher sollten kommende Arbeiten auf fortlaufende Evaluierung bestehen und auf mehr Resourcen zurückgreifen, um aussagekräftigere Schlüsse ziehen zu können. \\

Die Ergebnisse der Arbeit zeigen, dass die Anzahl der Iterationen wesentlichen Einfluss mit der Ergonomie und Zufriedenheit der Nutzer hat. Die weniger oft ausgeführten Evaluationszyklen - während der Konzeption der Wireframes - sorgte für Probleme bei der Erstellung der Gestaltungslösung in Form von Schleifen. Daher sollten die Evaluationen bereits, wie auch von der DIN Norm empfohlen, stets zu Beginn eingeplant und vor allem konsequent durchgeführt werden. \\

Aus den Ergebnissen lässt sich schließlich ableiten, dass die DIN Norm, für die Gestaltung von ergonomischen interaktiven Systemen sowie für den Austausch von kulturellen und traditionellen Rezepten geeignet ist, da die durchschnittliche Zufriedenheit der befragten Nutzer bei 8 von 10 Punkten lag. Die negativen Bewertungen von Teilsystemen ließen sich auf Mängel in der Durchführung des Prozesses oder Vorbereitung der Evaluierung zurückführen und belasten daher nicht die Bewertung der DIN Norm im Hinblick auf dieses System. \\

Somit wurde gezeigt, dass die DIN Norm für den Prozess zur Gestaltung gebrauchstauglicher interaktive Systeme geeignet ist. Sie sollte strikter befolgt werden, um Schleifen in den Arbeitsprozessen zu vermeiden beziehungsweise früher durchzuführen, um den Arbeitsaufwand zu reduzieren sowie die Zufriedenheit von Nutzern zu steigern.
Durch den Fokus auf die Gestaltung des Systems wurde im Rahmen dieser Projektarbeit nicht genauer auf die Auswahl der Frameworks für die Umsetzung eingegangen. Diese stellt jedoch einen bedeutenden Ansatz für zukünftige Arbeiten dar.
% wird fortlaufend auf der Basis benutzerzentrierter Evaluierung vorangetrieben
% Überblick über den Aufbau der Arbeit, Ergebnisse der einzelnen Kapitel
% Ergebnisse und Forschungsfrage in Beziehung setzen: „Harmonie zwischen den aus dem Thema abgeleiteten Fragestellung(en) und den im Schlussteil ausgewiesenen Ergebnissen, die Antworten zu diesen Fragestellungen geben“
% Ergebnisse in Forschungskontext einordnen, Geltungsbereich kritisch einschätzen (vgl. Winter 2004: 76); selbstkritische Reflexion, Kritikpunkte, Fehlstellen und Beschränkungen
% Schlussfolgerungen, offene Fragen (vgl. Samac, Prenner & Schwetz 2014: 74), Vorschläge für weitere Forschung: „future research“ (vgl. Franck 2004: 199)
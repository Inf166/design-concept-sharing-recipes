\chapter*{Kurzfassung}
\addcontentsline{toc}{chapter}{Kurzfassung}
% Problemstellung (These)
Durch die zunehmende Globalisierung erlebt die Gesellschaft den Verlust von Kultur und Tradition. Online Rezepte ersetzen Framilienrezepte. Hier angesetzt, soll durch ein ergonomisches, interaktives System zum Austausch und Weitergabe des Wissens und den Erfahrungen der Familie, das Problem des Kultur- und Traditionsverlusts adressiert und reduziert werden. Um die Weitergabe und den Austausch zu erleichtern, müssen die in diesem Gebiet klassischen Probleme untersucht und gelöst werden. Darunter fallen Probleme wie: der Import und Export, die Einheitlichkeit der Rezepte und damit gesteigerte Lesbarkeit, den Aufbau eines Kochbuchs, die kritische Auseinandersetzung mit Inhalten, aber auch die persönliche Verbindung zu dem gebündelten Wissen. Auf die Anlehnung an Kochbücher und gedruckte Rezepte, innerhalb der Gestaltungslösungen wird Wert gelegt. Der Schutz des Kulturguts einer Familie muss durch die Kommunikationswege, Nutzerverwaltung und Zugriffsrechtevergabe behandelt werden. Zusätzlich werden die Barrierefreiheit, Nachhaltigkeit, Instandhaltung, Skalierbarkeit, Erreichbarkeit und der Datenschutz untersucht und implementiert. Für die Gestaltung eines solchen Systems wird daher oft zu etablierten Gestaltungslösungen gegriffen.\\

% Zielsetzung (Was soll beantwortet werden)
Diese Arbeit beschäftigt sich damit, ob ein Vorgehen nach der DIN-EN-ISO 9421-210 die Zufriedenheit der Nutzer steigert. Dazu wird die folgende Forschungsfrage gestellt: \\
Wie zufriedenstellend ist ein System für den Austausch von traditionellen und kulturellen Rezepten, wenn es nach der DIN-EN-ISO 9421-210, für den Prozess zur Gestaltung gebrauchstauglicher interaktiver Systeme, konzipiert wird? \\

% Methodik (Was muss für die Antwort getan werden)
Um die Forschungsfrage zu beantworten, werden die wesentlichen Schritte der Norm durchlaufen und bearbeitet. Die Beteiligung einer repräsentativen Nutzergruppe an der Konzeption des Systems und der Priorisierung einzelner Anforderungen ist gewünscht und erforderlich. Die Anforderungen an das System werden mittels Interviews und gemeinsamer Erarbeitung von User Stories konkretisiert. Desweiteren werden diese Anforderungen priorisiert und dadurch Konflikte zwischen den Anforderungen beseitigt. \\

% Ergebnisse (Was stellte sich heraus)
Es wurde festgestellt, dass die Erfüllungsrate der Akzeptanzkriterien maßgeblich mit der Größe der evaluierenden Stakeholder korreliert. Die enge Zusammenarbeit mit diesen Stakeholdern ist daher unabdingbar. Die Evaluierungen der ersten Gestaltungslösungen hat die Gestaltung maßgeblich beeinflusst. Das Endergebnis macht deutlich, dass das Gestalten eines Interaktiven Systems nach der DIN-EN-ISO 9241-210 die Zufriedenheit gewährleistet und somit ein wichtiger Teil der Entwicklung sozialer Netzwerke zum Austausch von Wissen und Erfahrungen ist. \\

% Empfehlungen (Folge Projekte)
Die Entwickler von zukünftigen interaktiven Systemen, sollten sich mit der Zielgruppe befassen und sie in die Gestaltung mit einbeziehen, um die Systeme menschenzentrierter zu gestalten. Für die Weiterentwicklung des Datenschutzes sind weitere Forschungen im Zusammenspiel mit Progressiven Web Apps (PWA) und der Peer to Peer (P2P) Technologie notwendig, denn sie würden, wenn sie mehr etabliert sind und mehr Funktionen bieten, eine noch bessere Alternative zu bisherigen Umsetzungen solcher Systeme darstellen. \\


%\newpage
%\chapter*{Abstract}
%\addcontentsline{toc}{chapter}{Abstract}
%Hier folgt die Kurzfassung auf Englisch. Wenn Sie diese Vorlage für Seminararbeiten, Projektdokumentation o.ä. verwenden ist eine englische Kurzfassung ggf. nicht nötig.
